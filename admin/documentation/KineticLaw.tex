%% This is an evolving article style
%% steeling ideas from different places

\documentclass[12pt,english]{article}
\usepackage[T1]{fontenc}
\usepackage[latin1]{inputenc}
\usepackage{babel}
\usepackage[linktocpage,colorlinks,urlcolor=blue]{hyperref}

\newcommand{\boldsymbol}[1]{\mbox{\boldmath $#1$}}
\newcommand{\N}{\mbox{$I\!\!N$}}
\newcommand{\Z}{\mbox{$Z\!\!\!Z$}}
\newcommand{\Q}{\mbox{$I\:\!\!\!\!\!Q$}}
\newcommand{\R}{\mbox{$I\!\!R$}}

\makeindex 
\makeatletter
\makeatother

\begin{document}

\title{COPASI's and SBML's Kinetic Law Representation}

\author{Stefan Hoops\\
 Virginia Bioinformatics Institute - 0477\\
 Virginia Tech\\
 Bioinformatics Facility I\\
 Blacksburg, VA 24061 \\
 USA \\
 \href{mailto:Stefan Hoops <shoops@vt.edu>}{shoops@vt.edu}}

\date{2005-11-14}

\maketitle
\begin{abstract}
The presentation of kinetic laws in
\href{http://www.copasi.org}{COPASI} and
\href{http://prdownloads.sourceforge.net/sbml/sbml-level-2-v1.pdf?download}
{SBML Level 2 Version 1} differs. In this document we explain the
reason for this difference. In addition the
\href{http://www.sbml.org}{SBML} rate laws written by COPASI are known
to create problems in some software packages which are listed as
supporting SBML. We explain why we consider our presentation as
correct SBML and correct from the modeling standpoint.
\end{abstract}

\section{Rate Laws in COPASI}
Rate laws in COPASI are presented to the user as functions describing
the change of metabolites involved in a chemical reaction in
concentration per time. They are always dependent on the
concentrations of metabolites involved in the model. We are using this
representation since the user is familiar with it and the units for
kinetic constants found in the literature are normally consistent with
it.

However internally due to the fact that COPASI deals with multi
compartment models we have to calculate particle numbers and not
concentrations, since individual species items are preserved and not
their concentration. For example to correctly address the problem of
transport reactions between two compartments we must calculate the
change of the particle number of the metabolite in its compartment
instead of its change of concentration. This leads to the necessity to
reinterpret the rate laws.

This necessity is shown in the following example where we consider a
transport reaction from compartment $\boldsymbol{A}$ to compartment
$\boldsymbol{B}$. 
%
\begin{eqnarray}
\boldsymbol{S} \rightarrow \boldsymbol{P}
\end{eqnarray} 
%
Let the volumes of both compartment be $V_A = 2\, \mbox{l}$ and $V_B =
1\, \mbox{l}$. The rate law is simple irreversible Mass action with
constant $R = 1\, \frac{1}{\mbox{s}}$.  If we start with the initial
concentration of
%
\begin{eqnarray}
\left[\boldsymbol{S}\right]_0 = 1\; \mbox{Mol/l}
\mbox{,} &&
\left[\boldsymbol{P}\right]_0 = 0\; \mbox{Mol/l}
\end{eqnarray} 
%
By using the traditional rate law we have the following differential
equations 
%
\begin{eqnarray}
\frac{\delta \left[\boldsymbol{S}\right]}{\delta t} 
= - R \left[\boldsymbol{S}\right]
\mbox{,} &&
\frac{\delta \left[\boldsymbol{P}\right]}{\delta t} 
=   R \left[\boldsymbol{S}\right] 
\end{eqnarray} 
%
This leads to the solution
%
\begin{eqnarray}
\left[\boldsymbol{S}\right](t) = \left[\boldsymbol{S}\right]_0 \exp(-Rt)
\mbox{,} &&
\left[\boldsymbol{P}\right](t) = \left[\boldsymbol{S}\right]_0
(1 - \exp(-Rt))
\end{eqnarray} 
%
which for $t \rightarrow \infty$ leads to 
%
\begin{eqnarray}
\left[\boldsymbol{S}\right]_\infty = 0\; \mbox{Mol/l}
\mbox{,} &&
\left[\boldsymbol{P}\right]_\infty = \left[\boldsymbol{S}\right]_0 = 1\; \mbox{Mol/l} 
\end{eqnarray} 
%
This is clearly a contradiction to the mass conservation
$\boldsymbol{S} + \boldsymbol{P} = const.$ since the initial particle
number is $2\; \mbox{Mol}$ and the ending particle number is $1\;
\mbox{Mol}$.

To calculate the change of particle numbers instead of the change of
concentration we rescale the kinetic law for ordinary reactions ,
i.e., reactions involving only species in one compartment
automatically by the compartments volume. Reactions not involving
multiple compartments are not automatically rescaled and are
interpreted as describing changes of particle numbers per time. In the
above example this leads to the following differential equations
%
\begin{eqnarray}
\frac{\delta \boldsymbol{S}}{\delta t} 
= - R\, \left[\boldsymbol{S}\right]
\mbox{,} &&
\frac{\delta \boldsymbol{P}}{\delta t} 
=   R\, \left[\boldsymbol{S}\right] \
\end{eqnarray} 
%
which for constant volume is equivalent to
%
\hypertarget{COPASI_1}{
\begin{eqnarray}\label{COPASI_1}
\frac{\delta \left[\boldsymbol{S}\right]}{\delta t} 
= - \frac{R}{V_A} \left[\boldsymbol{S}\right]
\mbox{,} &&
\frac{\delta \left[\boldsymbol{P}\right]}{\delta t} 
=   \frac{R}{V_B} \left[\boldsymbol{S}\right] \
\end{eqnarray} 
}
%
With the above initial conditions
%
\begin{eqnarray}
\left[\boldsymbol{S}(t)\right] = \left[\boldsymbol{S}\right]_0 \, \exp(-Rt) 
\mbox{,} &&
\left[\boldsymbol{P}(t)\right] = \left[\boldsymbol{S}\right]_0 
\frac{V_A}{V_B}\, (1 - \exp(-Rt))
\end{eqnarray} 
%
which for $t \rightarrow \infty$ leads to 
%
\begin{eqnarray}
\left[\boldsymbol{S}\right]_\infty = 0\; \mbox{Mol/l}
\mbox{,} &&
\left[\boldsymbol{P}\right]_\infty
 = \left[\boldsymbol{S}\right]_0\, \frac{V_A}{V_B}
 = 2\; \mbox{Mol/l} 
\end{eqnarray} 
%
This result is in accordance with the mass conservation of species
$\boldsymbol{X}$.

Let us know look at the same reaction with $\boldsymbol{S}$ and
$\boldsymbol{P}$ in the same compartment with the volume
$V$. In COPASI we will have
%
\begin{eqnarray}
\frac{\delta \boldsymbol{S}}{\delta t} 
= - R\, V\, \left[\boldsymbol{S}\right]
\mbox{,} &&
\frac{\delta \boldsymbol{P}}{\delta t} 
=   R\, V\, \left[\boldsymbol{S}\right] \
\end{eqnarray} 
%
which for constant volume is equivalent to
%
\hypertarget{COPASI_2}{
\begin{eqnarray}\label{COPASI_2}
\frac{\delta \left[\boldsymbol{S}\right]}{\delta t} 
= - R \left[\boldsymbol{S}\right]
\mbox{,} &&
\frac{\delta \left[\boldsymbol{P}\right]}{\delta t} 
=   R \left[\boldsymbol{S}\right] \
\end{eqnarray} 
}
%
With the above initial conditions
%
\begin{eqnarray}
\left[\boldsymbol{S}(t)\right] = \left[\boldsymbol{S}\right]_0 \, \exp(-Rt) 
\mbox{,} &&
\left[\boldsymbol{P}(t)\right] = \left[\boldsymbol{S}\right]_0 
\, (1 - \exp(-Rt))
\end{eqnarray} 
%
which for $t \rightarrow \infty$ leads to expected result
%
\begin{eqnarray}
\left[\boldsymbol{S}\right]_\infty = 0\; \mbox{Mol/l}
\mbox{,} &&
\left[\boldsymbol{P}\right]_\infty
 = \left[\boldsymbol{S}\right]_0
 = 1\; \mbox{Mol/l} 
\end{eqnarray} 
%

\section{Rate Laws in SBML}
The rate law for reactions in SBML Level 2 Version 1 is explained in
section 4.9.7 of the specifications. Please note that paragraph 2
explicitly mentions that the rate law is not what is traditionally
understood. Instead the rate law describes the change of particle
numbers similarly to COPASI's internal representation. The actual
example given in the SBML specification is closely related the one
found above and the differential equations of the transport problem
described above would be in SBML notation
%
\hypertarget{SBML}{
\begin{eqnarray}\label{SBML}
\frac{\delta \left[\boldsymbol{S}\right]}{\delta t} 
= - \frac{R}{V_A} \left[\boldsymbol{S}\right]
\mbox{,} &&
\frac{\delta \left[\boldsymbol{P}\right]}{\delta t} 
=   \frac{R}{V_B} \left[\boldsymbol{S}\right] \
\end{eqnarray} 
}
%

\section{COPASI's rate laws in SBML}
As a result of the previous two chapters we see that for reactions
involving multiple compartments (\hyperlink{COPASI_1}{Equation
\ref*{COPASI_1}}) we can directly write the COPASI rate law as the
SBML rate (\hyperlink{SBML}{Equation \ref*{SBML}}) law. However, for
reactions which are involving only metabolites from one compartment
(\hyperlink{COPASI_2}{Equation \ref*{COPASI_2}}) we need to multiply
the rate law with the volume of that compartment which leads in our
example to
%
\begin{eqnarray}
R_{\mbox{\tiny{SBML}}} = V \, R_{\mbox{\tiny{COPASI}}}
\end{eqnarray} 
%
This equation seems to indicate that at least one of
$R_{\mbox{\tiny{SBML}}}$ or $R_{\mbox{\tiny{COPASI}}}$ is volume
dependent. Since the ODEs (\hyperlink{COPASI_2}{Equation
\ref*{COPASI_2}}) derived from the traditional rate laws lead to
concentration changes, which are independent from the volume of the
compartment for constant $R$ we have that $R_{\mbox{\tiny{COPASI}}}$
is volume independent, i.e, $R_{\mbox{\tiny{SBML}}}$ must be volume
dependent.

\end{document}
