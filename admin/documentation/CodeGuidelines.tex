\documentclass[12pt]{book}
\usepackage{graphicx}


\begin{document} 

\title{C++ Coding Guidelines \\ for \\ COPASI}
\author{Stefan Hoops\\
        Virginia Bioinformatics Institute \\ 1880 Pratt Dr. \\
        Blacksburg, VA 24060 \\ USA \\ shoops@vbi.vt.edu}

\maketitle

\tableofcontents

\chapter{Naming Conventions}
The intent of naming conventions is to allow programmers, which are
not familiar with the code to easily grasp the meaning and scope of
symbols in the source code. Each programmer of COPASI should adhere
for its own benefit and to the advantage of the project to the
following conventions.

\section{Class Names}
Class names must all start with a capital letter {\tt C}. This is
followed by a descriptive name. This name might be composed by
different words. These words must all start with capital letters and
are concatenated without underscores. Good examples for class names
include: 
  {\tt CCopasiXMLParser} , {\tt CExpatTemplate}, and {\tt CMathModel}.

\section{Variable Name}
In general a variable name should be descriptive. This name might be
composed by different words. These words must all start with capital
letters and are concatenated without underscores. In addition the
following standards should be followed:  

\begin{itemize}
\item {\bf Counters} might be used such as {\tt i}, {\tt k}, and {\tt
  l}, which may be used in loops.
\item {\bf Iterators} might be used such as {\tt it} and {\tt
  end}, which may be used in loops. 
\item {\bf Method Parameters} must start with a lower case letter.
\item {\bf Class Member Variables} are prepended with a lower case
  letter {\tt m}.
\item {\bf Pointers} are prepended with a lower case letter {\tt p}. 
\item {\bf Class Member Variables} which are {\bf Pointers} must have the
  prefix {\tt mp}.
\end{itemize}
 In addition the
following standards should be followed:  
\section{Method Name}
Method names should have a descriptive name starting with a lower case
letter. This name might be composed by different words. These words
beginning with the second must start with capital letters and
are concatenated without underscores. Good examples for method names
include: 
   {\tt createMetabolite}, {\tt compileIfNecessary}, and {\tt buildMoieties}

In addition the following standards should be followed:  

\begin{itemize}
\item {\bf Retrieval Methods} must start with {\tt get} followed by
  the member variable name without the prefix.
\item {\bf Set Methods} must start with {\tt set} followed by
  the member variable name without the prefix.
\item {\bf Boolean Query Functions} should start if applicable with
  {\tt is}.
\end{itemize}

\chapter{Program Code Guidelines}
\section{Loops}


\chapter{Code Documentation}
\section{Class Documentation}
\section{Variable Documentation}
\section{Method Documentation}
\section{Incline Code Documentation}

\chapter{Installation Structure}
This section defines the installation structure for COPASI on
different platforms. Each platform will adhere to the platform
specific requirements. 

\section{Unix}
The installation location needs to be available to COPASI at runtime
and therefore the environment variable {\tt COPASIDIR} pointing to this
location must be set by the user. 
{\tt \scriptsize
\begin{tabbing}
 \hspace{4 pt}\=\hspace{17 pt}\=\hspace{17 pt}\=\hspace{17
 pt}\=\hspace{17 pt}\=\hspace{17 pt}\= \\ [-12 pt]
 \$COPASIDIR \\
 \> +- bin \\
 \> | \> += CopasiUI \\
 \> +- share \\
 \> | \> +- copasi \\
 \> | \> \> +- doc \\
 \> | \> \> | \> +- html \\
 \> | \> \> | \> \> +- figures \\
 \> | \> \> | \> \> | \> +- DefaultPlotAdded.jpg \\
 \> | \> \> | \> \> | \> +- ModelSettingsDialog.jpg \\
 \> | \> \> | \> \> | \> +- ObjectBrowserSelection.jpg \\
 \> | \> \> | \> \> | \> +- ObjectBrowserTree.jpg \\
 \> | \> \> | \> \> | \> +- PlotDefinition.jpg \\
 \> | \> \> | \> \> | \> +- PlotWindow.jpg \\
 \> | \> \> | \> \> | \> +- ReactionDialog.jpg \\
 \> | \> \> | \> \> | \> +- ReactionOverview.jpg \\
 \> | \> \> | \> \> | \> +- ReactionOverviewEmpty.jpg \\
 \> | \> \> | \> \> | \> +- ReportDefinitionDialog.jpg \\
 \> | \> \> | \> \> | \> +- TimeCourseDialog.jpg \\
 \> | \> \> | \> \> +- TutWiz-Step1.html \\
 \> | \> \> | \> \> +- TutWiz-Step2.html \\
 \> | \> \> | \> \> +- TutWiz-Step3.html \\
 \> | \> \> | \> \> +- TutWiz-Step4.html \\
 \> | \> \> | \> \> +- TutWiz-Step5.html \\
 \> | \> \> | \> \> +- TutWiz-Step6.html \\
 \> | \> \> +- examples \\
 \> | \> \> \> +- CircadianClock.cps \\
 \> | \> \> \> +- Metabolism-2000Poo.xml \\
 \> | \> \> \> +- YeastGlycolysis.gps \\
 \> | \> \> \> +- brusselator.cps \\
 \> +- README \\
 \> +- ChangeLog \\
\end{tabbing}
}

\section{MacOS X}
The installation location must be available to COPASI at
runtime. However it is possible to determine the location through MacOS
X.
{\tt \scriptsize
\begin{tabbing}
 \hspace{4 pt}\=\hspace{17 pt}\=\hspace{17 pt}\=\hspace{17
 pt}\=\hspace{17 pt}\=\hspace{17 pt}\= \\ [-12 pt]
 \$COPASIDIR \\
 \> +- COPASI.app \\
 \> | \> +- Contents \\
 \> | \> +- MacOS \\
 \> | \> += CopasiUI \\
 \> | \> +- Resources \\
 \> | \> | \> +- doc \\
 \> | \> | \> | \> +- html \\
 \> | \> | \> | \> \> +- figures \\
 \> | \> | \> | \> \> | \> +- DefaultPlotAdded.jpg \\
 \> | \> | \> | \> \> | \> +- ModelSettingsDialog.jpg \\
 \> | \> | \> | \> \> | \> +- ObjectBrowserSelection.jpg \\
 \> | \> | \> | \> \> | \> +- ObjectBrowserTree.jpg \\
 \> | \> | \> | \> \> | \> +- PlotDefinition.jpg \\
 \> | \> | \> | \> \> | \> +- PlotWindow.jpg \\
 \> | \> | \> | \> \> | \> +- ReactionDialog.jpg \\
 \> | \> | \> | \> \> | \> +- ReactionOverview.jpg \\
 \> | \> | \> | \> \> | \> +- ReactionOverviewEmpty.jpg \\
 \> | \> | \> | \> \> | \> +- ReportDefinitionDialog.jpg \\
 \> | \> | \> | \> \> | \> +- TimeCourseDialog.jpg \\
 \> | \> | \> | \> \> +- TutWiz-Step1.html \\
 \> | \> | \> | \> \> +- TutWiz-Step2.html \\
 \> | \> | \> | \> \> +- TutWiz-Step3.html \\
 \> | \> | \> | \> \> +- TutWiz-Step4.html \\
 \> | \> | \> | \> \> +- TutWiz-Step5.html \\
 \> | \> | \> | \> \> +- TutWiz-Step6.html \\
 \> | \> | \> +- examples \\
 \> | \> | \> \> +- CircadianClock.cps \\
 \> | \> | \> \> +- Metabolism-2000Poo.xml \\
 \> | \> | \> \> +- YeastGlycolysis.gps \\
 \> | \> | \> \> +- brusselator.cps \\
 \> | \> +- Info.plist \\
 \> | \> +- ChangeLog \\
 \> +- COPASI-README.rtf \\
\end{tabbing}
}

\section{Windows}
The installation location must be available to COPASI at
runtime. However it is possible to determine the location through
Windows specific means.
{\tt \scriptsize
\begin{tabbing}
 \hspace{4 pt}\=\hspace{17 pt}\=\hspace{17 pt}\=\hspace{17
 pt}\=\hspace{17 pt}\=\hspace{17 pt}\= \\ [-12 pt]
 \$COPASIDIR \\
 \> +- bin \\
 \> | \> += CopasiUI \\
 \> +- share \\
 \> | \> +- copasi \\
 \> | \> \> +- doc \\
 \> | \> \> | \> +- html \\
 \> | \> \> | \> \> +- figures \\
 \> | \> \> | \> \> | \> +- DefaultPlotAdded.jpg \\
 \> | \> \> | \> \> | \> +- ModelSettingsDialog.jpg \\
 \> | \> \> | \> \> | \> +- ObjectBrowserSelection.jpg \\
 \> | \> \> | \> \> | \> +- ObjectBrowserTree.jpg \\
 \> | \> \> | \> \> | \> +- PlotDefinition.jpg \\
 \> | \> \> | \> \> | \> +- PlotWindow.jpg \\
 \> | \> \> | \> \> | \> +- ReactionDialog.jpg \\
 \> | \> \> | \> \> | \> +- ReactionOverview.jpg \\
 \> | \> \> | \> \> | \> +- ReactionOverviewEmpty.jpg \\
 \> | \> \> | \> \> | \> +- ReportDefinitionDialog.jpg \\
 \> | \> \> | \> \> | \> +- TimeCourseDialog.jpg \\
 \> | \> \> | \> \> +- TutWiz-Step1.html \\
 \> | \> \> | \> \> +- TutWiz-Step2.html \\
 \> | \> \> | \> \> +- TutWiz-Step3.html \\
 \> | \> \> | \> \> +- TutWiz-Step4.html \\
 \> | \> \> | \> \> +- TutWiz-Step5.html \\
 \> | \> \> | \> \> +- TutWiz-Step6.html \\
 \> | \> \> +- examples \\
 \> | \> \> \> +- CircadianClock.cps \\
 \> | \> \> \> +- Metabolism-2000Poo.xml \\
 \> | \> \> \> +- YeastGlycolysis.gps \\
 \> | \> \> \> +- brusselator.cps \\
 \> +- README \\
 \> +- ChangeLog \\
\end{tabbing}
}

\section{Handling Installation Differences}
The handling of differences in the installation structure must be
dealt with in one place within the COPASI code. The place for this is
the class {\tt COptions}. In this class the method:
{\tt \scriptsize
\begin{tabbing}
template< class CType > static void getValue(\=const std::string \&
name, \\
\> CType \& value)
\end{tabbing}
}
\noindent
provides access to common options within COPASI. The following values
will deal with installation dependent settings:
{\tt CopasiDir}, {\tt ExampleDir}, and {\tt WizardDir}. The following code
shows howe to retrieve the location of the examples files for COPASI: \\
{\tt \scriptsize
\indent std::string ExampleDir; \\[-4 pt]
\indent COptions::getValue(``ExampleDir'', ExampleDir); \\
}
\noindent
To assure that the values are correctly set any main program must
call: \\
{\tt \scriptsize
\indent       COptions::init(argc, argv);
}


\end{document}
